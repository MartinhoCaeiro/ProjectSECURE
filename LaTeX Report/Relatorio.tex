\documentclass[a4paper]{article}
% Pacotes necessários
\usepackage[portuguese]{babel}
\usepackage[backend=biber, style=apa, citestyle=apa, language=portuguese]{biblatex}
\usepackage{csquotes}
\addbibresource{Recursos/referencias.bib}

\usepackage{amsmath}
\usepackage{graphicx}
\usepackage{subcaption}
\usepackage{setspace}
\usepackage{siunitx} % Required for alignment
\sisetup{
  round-mode          = places, % Rounds numbers
  round-precision     = 2, % to 2 places
}
\usepackage{enumerate}
\usepackage{enumitem}
\usepackage{amsmath}
\usepackage{karnaugh-map}
\usepackage[section]{placeins}
\usepackage{geometry}
\usepackage{amssymb}
\usepackage{titling}
\usepackage[T1]{fontenc}
\usepackage{float}
\usepackage[hidelinks]{hyperref}
\usepackage{xcolor}
\usepackage{indentfirst}
\usepackage{array}
\usepackage{wrapfig} % Coloca isto no preâmbulo
\usepackage{soul}
\usepackage{afterpage}
\usepackage[toc,page]{appendix}
\newcolumntype{P}[1]{>{\centering\arraybackslash}p{#1}}
\onehalfspacing


% Comando para criar uma página vazia
\newcommand\myemptypage{
    \null
    \thispagestyle{empty}
    \addtocounter{page}{-1}
    \newpage
}

% Página de título principal
\newcommand{\firsttitlepage}{
    \begin{titlepage}
        \centering
        \vspace*{1cm}
        
        % Logos superior
        \begin{figure}[h!]
            \centering
            \includegraphics[width=6cm]{Recursos/Logos/LOGO_IPB} % Substitua pelo caminho da imagem
            \vspace{0.5cm}
        \end{figure}

        % Informações da instituição
        \large\textbf{INSTITUTO POLITÉCNICO DE BEJA} \\
        \large\textbf{Escola Superior de Tecnologia e Gestão} \\
        \large\textbf{Licenciatura em Engenharia Informática} \\
        \large\textbf{Projeto Final de Curso} \\
        
        \vspace{2cm}
        
        % Título do projeto
        {\Huge \textbf{Desenvolvimento de um sistema de comunicações seguras }} \\
        
        \vspace{1cm}
        
        % Autores
    
        \large Martinho José Novo Caeiro - 23917 \\
        \large Paulo António Tavares Abade - 23919 \\
        \large Rafael Conceição Narciso - 24473 \\
        
        \vfill
        
        % Logo inferior
        \begin{figure}[h!]
            \centering
            \includegraphics[width=6cm]{Recursos/Logos/IPBejaESTIG.jpg} % Substitua pelo caminho da imagem
        \end{figure}
        
        % Local e data
        {\large Beja, março de 2025}
    \end{titlepage}
}

\newcommand{\secondtitlepage}{
    \begin{titlepage}
        \centering
        \vspace*{1cm}
        
        % Informações da instituição
        \large\textbf{INSTITUTO POLITÉCNICO DE BEJA} \\
        \large\textbf{Escola Superior de Tecnologia e Gestão} \\
        \large\textbf{Licenciatura em Engenharia Informática} \\
        \large\textbf{Projeto Final de Curso} \\
        
        \vspace{2cm}
        
        % Título do projeto
        {\Huge \textbf{Desenvolvimento de um sistema de comunicações seguras }} \\
        
        \vspace{1.5cm}
        
        % Autores
        \large Martinho José Novo Caeiro - 23917 \\
        \large Paulo António Tavares Abade - 23919 \\
        \large Rafael Conceição Narciso - 24473 \\

        \vspace{2cm}

        % Orientador
        \large Orientador: Professor Rui Silva \\
        
        \vfill
        
        % Local e data
        {\large Beja, março de 2025}
    \end{titlepage}
}

\begin{document}


\pagenumbering{gobble} % Oculta numeração da página

% Primeira página de título
\firsttitlepage

\secondtitlepage


% Abstract
\section*{\LARGE\textbf{\textit{Resumo}}}

Neste relatório será abordado o processo de criação de uma solução de comunicações seguras, 
que permita a troca de mensagens de texto entre os seus utilizadores.
Este relatório foi realizado no âmbito da Unidade Curricular de
Estágio ou Projeto (\cite{pagep}).


\vspace{1cm}
% Keywords
\textbf{Keywords:} aplicações, cibersegurança, comunicações, criptografia
\newpage
%--------------------------------------------------------------------------------------------------------------------------------------

\section*{\LARGE\textbf{\textit{Abstract}}}

In this report, we will address the creation process of a secure communication 
solution that allows text message exchange between its users.
This report was carried out within the scope of the 
Curricular Unit of Internship or Project (\cite{pagep}).


\vspace{1cm}
% Keywords
\textbf{Keywords:} applications, cybersecurity, communications, cryptography
\renewcommand{\contentsname}{Índice}       % Título do sumário
\renewcommand{\listfigurename}{Índice de Figuras} % Título da lista de figuras

% Início do conteúdo do relatório
\newpage
\doublespacing
\tableofcontents
\newpage
\listoffigures
\doublespacing

\newpage
\pagenumbering{arabic}

\section{Introdução}\label{intro}
Para a realização do projeto é necessário desenvolver um sistema de comunicações seguras, que permita a troca de mensagens de texto entre os seus utilizadores.
Para tal, é necessário implementar um sistema de autenticação de utilizadores, que permita a criação de contas de utilizador e a autenticação dos mesmos.
O sistema deve ser capaz de garantir a confidencialidade, integridade e autenticidade das mensagens trocadas entre os utilizadores.

%---------------------------------------------------------------------------------------------------------------------------
\newpage
\section{Analise de Requisitos}\label{analise}
Para a realização deste projeto, foram necessários os seguintes requisitos:
\begin{itemize}
    \item \textbf{Requisitos Funcionais:}
    \begin{itemize}
        \item O sistema deve permitir a criação de contas de utilizador.
        \item O sistema deve permitir a autenticação de utilizadores.
        \item O sistema deve permitir o envio e receção de mensagens entre utilizadores.
        \item O sistema deve garantir a confidencialidade, integridade e autenticidade das mensagens trocadas.
    \end{itemize}
    \item \textbf{Requisitos Não Funcionais:}
    \begin{itemize}
        \item O sistema deve ser seguro e resistente a ataques.
        \item O sistema deve ser fácil de usar e intuitivo.
        \item O sistema deve ser escalável e capaz de suportar um grande número de utilizadores.
    \end{itemize}
\end{itemize}

%---------------------------------------------------------------------------------------------------------------------------
\newpage
\section{Tecnologias Utilizadas}\label{tecnologias}
Para o desenvolvimento deste projeto, foram utilizadas as seguintes tecnologias:
\begin{itemize}
    \item \textbf{Linguagens de Programação:} Python, JavaScript
    \item \textbf{Frameworks:} Flask, React
    \item \textbf{Banco de Dados:} SQLite
    \item \textbf{Criptografia:} PyCryptodome, hashlib
    \item \textbf{Ferramentas de Desenvolvimento:} Git, Visual Studio Code, Android Studio
    \item  \textbf{Serviços de Hospedagem:} Heroku, GitHub
\end{itemize}

%---------------------------------------------------------------------------------------------------------------------------
\newpage
\section{Arquitetura do Sistema}\label{arquitetura}

%---------------------------------------------------------------------------------------------------------------------------
\newpage
\section{Modulos do Sistema}\label{modulos}

%---------------------------------------------------------------------------------------------------------------------------
\newpage
\section{Desenvolvimento}\label{desenvolvimento}

%---------------------------------------------------------------------------------------------------------------------------
\newpage
\section{Testes}\label{testes}

%---------------------------------------------------------------------------------------------------------------------------
\newpage
\section{Conclusão}\label{con}
Concluimos assim que o projeto foi um sucesso, pois conseguimos desenvolver um sistema de comunicações seguras
que permite a troca de mensagens de texto entre os seus utilizadores, sem que haja a possibilidade de terceiros acederem às mensagens trocadas.
Em geral, foi um projeto que nos permitiu aprender bastante sobre a área de cibersegurança e comunicações seguras,
e que nos possibilitou aplicar os conhecimentos adquiridos ao longo do curso de Engenharia Informática.
%---------------------------------------------------------------------------------------------------------------------------

\newpage
\renewcommand{\refname}{Bibliografia} % Para artigos
\renewcommand{\bibname}{Bibliografia} % Para livros e relatórios
\addcontentsline{toc}{section}{Bibliografia} % Adiciona a Bibliografia ao índice
\printbibliography
\newpage
\end{document}
